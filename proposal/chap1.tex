%
% This is Chapter 1 file (chap1.tex)
%
\chapter{Introduction}
The past decades have witnessed the tremendous success of World Wide Web. 
People all over the world can now access to publicly available 
information via commercial search engines such as Google, Microsoft Bing 
with great ease. According to the online statistics \footnote{http://www.internetlivestats.com/google-search-statistics/}, 
Google now (as of October 2016) can handle over 40,000 search queries 
every second on average, which translates to over 3.5 billion searches 
per day and 1.2 trillion searches per year worldwide. 
With such huge volume of search activities it is essential to make the 
search results of high quality in order to meet the users needs.

Information Retrieval (IR), usually used by academia in favor of its 
industrial counterpart search engine, 
is one of the most evolving fields and has drawn extensive attention in 
recent years. 
Besides the search quality which is the most direct goal of IR, there 
are several other aspects which are also crucial to the success of IR 
system. For example, the evaluation streamline is one of the key 
components of IR system where different ranking approaches can be 
easily compared. Moreover, there are many other web applications which 
are highly related to the IR system. One of such domain is recommendation 
system where researchers tried their best to incorporate IR techniques 
with this area hoping to satisfy users different needs.


The primary goal of IR research is to effectively address user's 
information needs such as search via text queries or recommendation 
based on historical activities. 

%The word ``\textit{Context}'' is originally defined as 
%``the set of circumstances or facts that surround a particular event, situation, etc.'' 