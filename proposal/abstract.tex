%
% This is Abstract
%
Information Retrieval (IR) is one of the most evolving research fields 
and has drawn extensive attention in recent years. 
How to effectively address users information needs in one of the most 
important goal of IR research. This lays in the improvement of ranking 
models on top of traditional approaches. Besides this goal, there 
are several other aspects which are also crucial to the success of IR 
system. For example, the evaluation streamline is one of the key 
components of IR system where different ranking approaches can be 
easily compared and evaluated. 
Moreover, there are many other web applications which 
are highly related to the IR system. One of such domain is recommendation 
system where researchers tried their best to incorporate IR techniques 
hoping to satisfy different users needs.

The word ``\textit{Context}'' is originally defined as ``the set of circumstances or facts that surround a particular event, situation, etc.'' 
Previous studies in IR seldom treat the context as a separated yet 
crucial part. But context indeed plays an very important role and it 
is even the basis of the aforementioned IR research.
For example, classic IR ranking models are mainly based on 
bag-of-terms document representation assumption and they mainly 
consist of Term Frequency, Inverted Document Frequency, Document Length 
Normalization and other statistics. Here the bag-of-terms assumption 
and the commonly used statistics are the context if one's goal is to 
improve the effectiveness of bag-of-terms ranking models. 
For evaluation of IR system the context would be the unified testing 
environment where all the ranking models are judged purely based on 
their method/algorithm and the results are standardized. Here the unified 
testing environment is the context of the evaluation process. 
When dealing with places/venues recommendation the context would be the 
user's geographic location, temporal information and other related 
information. 

In this proposal, we identify the context of three different domains 
of IR researches and investigate the impact of the contexts.  

