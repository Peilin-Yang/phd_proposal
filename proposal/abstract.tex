%
% This is Abstract
%
Information Retrieval (IR) is one of the most evolving research fields 
and has drawn extensive attention in recent years. 
There are different types of IR systems targeting to meet 
different users' information needs. Two IR systems are of the most interest: (1) the contextual suggestion system that shows interesting venues to end users, (2) the evaluation system for IR researchers who are 
interested in understanding the IR ranking models.
In order to make better IR systems that are satisfied by both researchers 
and end users, this proposal explorers the impact of 
context in two specific long-term IR related tasks -- contextual suggestion 
and ranking model performance upper bound analysis using unified evaluation 
system.

For contextual suggestion system it is commonly agreed that successful 
modeling user profile is the key to the solution. As proposed here, user's 
long-term preference should also be considered as part of the context of 
this problem. 
The first approach is to model the user profile using the venue's category 
and description from user's activity history. This proposal further 
improves the 
method by leveraging the opinions from user's preference to model the 
user profile. Such methodology utilizes the rich text associated 
with the users which naturally fits the problem into IR domain. 
By modeling the candidate suggestions in the similar fashion, 
the similarities between the candidates and the user profile are used to 
score the candidates. Experiments on TREC collections and a Yelp data 
collection confirm the advantage of the method.

A long standing problem of IR community is that newly proposed ranking models 
are implemented and evaluated using various systems of researchers' choices 
and such choices lead to the fact that different researchers reported 
different results for the same baseline model. 
The context -- the IR evaluation system used by researcher, 
is apparently the cause of such problem. This fact casts doubts on the real 
effectiveness of the proposed ranking models and it is essential for the IR 
community to have a unified evaluation system so that ranking models can be 
compared without considering the details -- usually other components of an 
IR system, e.g. preprocessing of documents or the choice of evaluation 
metrics. 
This proposal proposes two unified IR evaluation systems to alleviate this problem. 
VIRLab as the first attempt provides a web based IR evaluation system 
where the users of the system can implement their own models and get them 
automatically evaluated.
Another system RISE (Reproducible IR System Evaluation) is proposed in order 
to further improve VIRLab.
RISE is the first ever known system that users of the system can collaborate 
with each other and thus make the normalization of baseline models much more 
easier and thus can be trusted by future researchers. 
With the help of RISE a large scale reproducibility experiment is deployed 
and the comprehensive results serve as the the performances reference of 
most widely used IR models.

Based on the reproducibility experiment results of using RISE it seems that 
the optimum performances of bag-of-terms ranking models do not improve a lot 
on TREC collections for over 20 years. 
This introduces another interesting question: does it exist the performance 
upper bound for bag-of-terms models? To answer this question an extensive 
investigation of the context -- the bag-of-terms document representation 
assumption and the statistics like term frequency and inverted document 
frequency with which the models usually play.
For single term queries several ranking models can be transformed to a 
simplified model. The cost/gain analysis which is commonly used in 
learning-to-rank (LTR) domain is then applied in order to find the optimum 
of single term queries for this simplified model.
The result shows that although the performances of state-of-the-art 
ranking models are quit close to the practical optimum there is still some 
room for improvement.

For the future work, there are mainly two directions to explore more deeper. 
The first one is to quantify the impact of the context of unified 
IR evaluation system. There is no previous work on how much difference does 
the usage of different retrieval tools bring.
The hope is to be the first to report on standardizing and quantifying the 
impact so that the IR community could be aware of such divergence and can 
better evaluate the contributions of using various tools. 
The second one is to provide more sound justification about the boundary 
theory of ranking model performance. Specifically, the cost/gain 
analysis could be extended to to multiple term queries. 
Other methods could also be tried, e.g. explanatory analysis, to achieve 
the same goal.
